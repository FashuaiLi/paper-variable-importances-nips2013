\documentclass[final]{beamer}

\usetheme{RJH}
\usepackage[utf8]{inputenc}
\usepackage[frenchb]{babel}
\usepackage[orientation=landscape,size=a2,scale=1.4,debug]{beamerposter}
\usepackage[absolute,overlay]{textpos}
\usepackage{url}
\usepackage{auto-pst-pdf}
\usepackage{pst-plot}
\usepackage{graphicx}
\usepackage{color}
\usepackage{hyperref}
\usepackage{amsmath}

\setlength{\TPHorizModule}{\paperwidth}
\setlength{\TPVertModule}{\paperheight}

\title{Understanding variables importances in forests of randomized trees}
\author{Gilles Louppe, Louis Wehenkel, Antonio Sutera and Pierre Geurts}

\footer{\tiny \textbf{Contact}: \url{g.louppe@ulg.ac.be} $\cdot$ \href{https://twitter.com/glouppe}{@glouppe} \textbf{Code}: \url{https://github.com/glouppe/paper-variable-importances}}
\date{}

\definecolor{lightgreen}{rgb}{0.094,0.737,0.611}
\definecolor{lightblue}{rgb}{0.137,0.513,0.768}
\definecolor{lightred}{rgb}{0.874,0.180,0.105}
\definecolor{gray}{rgb}{0.5,0.5,0.5}

\begin{document}
\begin{frame}{}


%% Column 1 ==================================================================

\begin{textblock}{0.32}(0.01,0.14)

%% Abstract ------------------------------------------------------------------

\begin{block}{Abstract \phantom{p}}

Despite growing interest and practical use in various scientific areas, \textbf{variable
importances derived from tree-based ensemble methods are not well understood
from a theoretical point of view}. In this work we characterize the Mean Decrease
Impurity (MDI) variable importances as measured by an ensemble of totally
randomized trees in asymptotic sample and ensemble size conditions. \textbf{We derive a
three-level decomposition of the  information jointly provided by all input
variables about the output in terms of {\color{lightgreen}i) the MDI importance of each input
variable}, {\color{lightblue}ii) the degree of interaction of a given input variable with the other
input variables}, {\color{lightred}iii) the different interaction terms of a given degree}}. We then
show that this MDI importance of a variable is equal to zero if and only if the
variable is irrelevant and that the MDI importance of a relevant variable is
invariant with respect to the removal or the addition of irrelevant variables.
We illustrate these properties on a simple example and discuss how they may
change in the case of  non-totally randomized trees such as Random Forests and
Extra-Trees.

\end{block}

\begin{block}{Variable importances in trees \phantom{p}}

\textbf{Notations.} Let assume a set $V = \{X_1, ..., X_p\}$ of categorical input variables and a categorical output variable $Y$. Given a training sample ${\cal L}$ of $N$ joint observations of $X_1, ..., X_p, Y$ drawn from $P(X_1, ..., X_p, Y)$,
let us define for any internal node $t$ of a decision tree built from ${\cal L}$:
\begin{itemize}
\item[-] The number of training samples in $t$ as $N_t$;
\item[-] The proportion of training samples in $t$ as $p(t) = \frac{N_t}{N}$;
\item[-] The impurity of node $t$ as $i(t) = H(Y|t)$ (i.e., the Shannon entropy);
\item[-] The impurity decrease at node $t$ as $\Delta i(t) = i(t) - \frac{N_{t_L}}{N_t} i(t_L) - \frac{N_{t_R}}{N_t} i(t_R)$.
\end{itemize}

\begin{center}

\scalebox{1.2}{
    \begin{pspicture}(14,11)
    % Grid
    %\psgrid[subgriddiv=1,griddots=10,gridlabels=7pt]
    % Calcul
    \rput[l](9,9.5){\color{gray}{$i(t) = 0.97$}}
    \rput[l](9,8.5){\color{gray}{$i(t_L) = 0.65$}}
    \rput[l](9,7.5){\color{gray}{$i(t_R) = 0.81$}}
    \rput[l](9,6.5){\color{gray}{$\Delta i(t) = i(t) - \frac{12}{20} i(t_L) - \frac{8}{20} i(t_R)$}}
    \rput[l](10.55,5.5){\color{gray}{$= 0.25$}}
    % Parent node
        % Arrows
        \psline[linewidth=2pt,linecolor=gray]{->}(4.5,10.5)(5,10)
        \psline[linewidth=2pt,linecolor=gray]{->}(5,9)(3,7)
        \psline[linewidth=2pt,linecolor=gray]{->}(5,9)(7,7)
        % Node
        \pscircle[fillstyle=solid,fillcolor=white,linewidth=2pt](5,9){1}
        % Samples
        \psframe[fillstyle=solid,fillcolor=red,linecolor=red](4.3,9.35)(4.5,9.55)
        \psframe[fillstyle=solid,fillcolor=lightblue,linecolor=lightblue](4.6,9.35)(4.8,9.55)
        \psframe[fillstyle=solid,fillcolor=lightblue,linecolor=lightblue](4.9,9.35)(5.1,9.55)
        \psframe[fillstyle=solid,fillcolor=lightblue,linecolor=lightblue](5.2,9.35)(5.4,9.55)
        \psframe[fillstyle=solid,fillcolor=lightblue,linecolor=lightblue](5.5,9.35)(5.7,9.55)
        \psframe[fillstyle=solid,fillcolor=lightblue,linecolor=lightblue](4.3,9.05)(4.5,9.25)
        \psframe[fillstyle=solid,fillcolor=red,linecolor=red](4.6,9.05)(4.8,9.25)
        \psframe[fillstyle=solid,fillcolor=red,linecolor=red](4.9,9.05)(5.1,9.25)
        \psframe[fillstyle=solid,fillcolor=lightblue,linecolor=lightblue](5.2,9.05)(5.4,9.25)
        \psframe[fillstyle=solid,fillcolor=lightblue,linecolor=lightblue](5.5,9.05)(5.7,9.25)
        \psframe[fillstyle=solid,fillcolor=lightblue,linecolor=lightblue](4.3,8.75)(4.5,8.95)
        \psframe[fillstyle=solid,fillcolor=red,linecolor=red](4.6,8.75)(4.8,8.95)
        \psframe[fillstyle=solid,fillcolor=lightblue,linecolor=lightblue](4.9,8.75)(5.1,8.95)
        \psframe[fillstyle=solid,fillcolor=lightblue,linecolor=lightblue](5.2,8.75)(5.4,8.95)
        \psframe[fillstyle=solid,fillcolor=red,linecolor=red](5.5,8.75)(5.7,8.95)
        \psframe[fillstyle=solid,fillcolor=lightblue,linecolor=lightblue](4.3,8.45)(4.5,8.65)
        \psframe[fillstyle=solid,fillcolor=lightblue,linecolor=lightblue](4.6,8.45)(4.8,8.65)
        \psframe[fillstyle=solid,fillcolor=red,linecolor=red](4.9,8.45)(5.1,8.65)
        \psframe[fillstyle=solid,fillcolor=red,linecolor=red](5.2,8.45)(5.4,8.65)
        \psframe[fillstyle=solid,fillcolor=red,linecolor=red](5.5,8.45)(5.7,8.65)
        % Text
        \rput(6,10){$t$}
    % Left node
        % Arrows
        \psline[linewidth=2pt,linecolor=gray](3,6)(2,5)
        \psline[linewidth=2pt,linecolor=gray](3,6)(4,5)
        % Node
        \pscircle[fillstyle=solid,fillcolor=white,linewidth=2pt](3,6){1}
        % Samples
        \psframe[fillstyle=solid,fillcolor=lightblue,linecolor=lightblue](2.3,6.35)(2.5,6.55)
        \psframe[fillstyle=solid,fillcolor=lightblue,linecolor=lightblue](2.6,6.35)(2.8,6.55)
        \psframe[fillstyle=solid,fillcolor=lightblue,linecolor=lightblue](2.9,6.35)(3.1,6.55)
        \psframe[fillstyle=solid,fillcolor=lightblue,linecolor=lightblue](3.2,6.35)(3.4,6.55)
        \psframe[fillstyle=solid,fillcolor=lightblue,linecolor=lightblue](3.5,6.35)(3.7,6.55)
        \psframe[fillstyle=solid,fillcolor=lightblue,linecolor=lightblue](2.3,6.05)(2.5,6.25)
        \psframe[fillstyle=solid,fillcolor=lightblue,linecolor=lightblue](2.6,6.05)(2.8,6.25)
        \psframe[fillstyle=solid,fillcolor=red,linecolor=red](2.9,6.05)(3.1,6.25)
        \psframe[fillstyle=solid,fillcolor=lightblue,linecolor=lightblue](3.2,6.05)(3.4,6.25)
        \psframe[fillstyle=solid,fillcolor=lightblue,linecolor=lightblue](3.5,6.05)(3.7,6.25)
        \psframe[fillstyle=solid,fillcolor=lightblue,linecolor=lightblue](2.3,5.75)(2.5,5.95)
        \psframe[fillstyle=solid,fillcolor=red,linecolor=red](2.6,5.75)(2.8,5.95)
        % Text
        \rput(4,7){$t_L$}
        \rput(2.5,8){$X_m = 0$}
    % Right node
        % Arrows
        \psline[linewidth=2pt,linecolor=gray](7,6)(6,5)
        \psline[linewidth=2pt,linecolor=gray](7,6)(8,5)
        % Node
        \pscircle[fillstyle=solid,fillcolor=white,linewidth=2pt](7,6){1}
        % Samples
        \psframe[fillstyle=solid,fillcolor=red,linecolor=red](6.3,6.35)(6.5,6.55)
        \psframe[fillstyle=solid,fillcolor=red,linecolor=red](6.6,6.35)(6.8,6.55)
        \psframe[fillstyle=solid,fillcolor=lightblue,linecolor=lightblue](6.9,6.35)(7.1,6.55)
        \psframe[fillstyle=solid,fillcolor=red,linecolor=red](7.2,6.35)(7.4,6.55)
        \psframe[fillstyle=solid,fillcolor=lightblue,linecolor=lightblue](7.5,6.35)(7.7,6.55)
        \psframe[fillstyle=solid,fillcolor=red,linecolor=red](6.3,6.05)(6.5,6.25)
        \psframe[fillstyle=solid,fillcolor=red,linecolor=red](6.6,6.05)(6.8,6.25)
        \psframe[fillstyle=solid,fillcolor=red,linecolor=red](6.9,6.05)(7.1,6.25)
        % Text
        \rput(8,7){$t_R$}
        \rput(7.5,8){$X_m = 1$}
    \end{pspicture}
}

\end{center}

\textbf{Definition.} In an ensemble of decision trees, the \textit{Mean
Decrease Impurity} (MDI) importance of an input variable $X_m$ is the
sum of the weighted impurity decreases $p(t)\Delta i(t)$, for all nodes $t$
where $X_m$ is used, averaged over all $N_T$ trees in the ensemble:
\begin{equation*}\label{eq:mdi}
Imp(X_m) = \frac{1}{N_T} \sum_{T} \sum_{t \in T:v(s_t) = X_m} p(t) \Delta i(t)
\end{equation*}

\end{block}



\end{textblock}



%% Column 2 ==================================================================

\begin{textblock}{0.32}(0.34,0.14)

\begin{block}{Theoretical results \phantom{p}}

\textbf{Theorem 1.}
\textit{The MDI importance of $X_m \in V$ for $Y$ as computed
with an   infinite ensemble of fully developed totally randomized trees and an
infinitely large training sample is:
  \begin{equation*}
  Imp(X_m)=\underbrace{\sum_{k=0}^{p-1} \frac{1}{C_p^k} \frac{1}{p-k}}_{{\color{lightblue} \substack{\text{ii) Decomposition along}\\
                                                                                                     \text{the degrees $k$ of interaction}\\
                                                                                                     \text{with the other variables}}}}
           \underbrace{\sum_{B \in {\cal P}_k(V^{-m})} I(X_m;Y|B)}_{{\color{lightred} \substack{\text{iii) Decomposition along all}\\
                                                                                                \text{interaction terms $B$}\\
                                                                                                \text{of a given degree $k$}}}},
  \end{equation*}
\noindent where $V^{-m}$ denotes the subset $V \setminus \{X_m\}$, ${\cal
P}_k(V^{-m})$ is the set of subsets of  $V^{-m}$ of cardinality $k$, and
$I(X_m;Y|B)$ is the conditional mutual information of $X_{m}$ and $Y$ given the
variables in $B$.}

\vspace{0.5cm}

\textbf{Theorem 2.}
\textit{For any ensemble of fully developed trees in asymptotic learning sample size
conditions, we have that}
\begin{equation*}
\underbrace{\sum_{m=1}^{p}Imp(X_m)}_{{\color{lightgreen} \substack{\text{i) Decomposition in terms of}\\
                                                                   \text{the MDI importance of}\\
                                                                   \text{each input variable}}}} =
\underbrace{I(X_{1}, \ldots, X_{p} ; Y)}_{\substack{\text{Information jointly provided}\\
                                                    \text{by all input variables}\\
                                                    \text{about the output}}}.
\end{equation*}

\vspace{0.5cm}

\textbf{Theorem 3.}
\textit{$X_i \in V$ is irrelevant to $Y$ with respect to $V$ if and only if  its
infinite sample size importance as computed with an infinite ensemble of fully
developed totally randomized trees built on $V$ for $Y$ is 0.}

\vspace{0.5cm}

\textbf{Theorem 5.}
\textit{Let $V_R \subseteq V$ be the subset of all variables in $V$ that are relevant to $Y$ with
respect to $V$. The infinite sample size importance of any variable $X_m \in
V_R$ as computed with an infinite ensemble of fully developed totally randomized
trees built on $V_R$ for $Y$ is the same as its importance computed in the same conditions by using all variables in $V$.}

\end{block}

\begin{block}{Non-totally randomized trees \phantom{p}}

\end{block}

\end{textblock}


%% Column 3 ==================================================================

\begin{textblock}{0.32}(0.67,0.14)

\begin{block}{Illustration \phantom{p}}

\end{block}

\begin{block}{Conclusions \phantom{p}}

\end{block}

\end{textblock}




\end{frame}
\end{document}
