\documentclass[final]{beamer}

\usetheme{RJH}
\usepackage[utf8]{inputenc}
\usepackage[frenchb]{babel}
\usepackage{verbatim}
\usepackage{graphicx}
\usepackage{color}
\usepackage{hyperref}
\usepackage{verbatim}
\usepackage{url}
\usepackage{auto-pst-pdf}
\usepackage{pst-plot}
\usepackage{pifont}
\newcommand{\cmark}{\ding{51}}%
\newcommand{\xmark}{\ding{55}}%

\hypersetup{colorlinks=true, linkcolor=black, urlcolor=blue}
\beamertemplatenavigationsymbolsempty
\setbeamertemplate{sections/subsections in toc}[circle]

\definecolor{lightgreen}{rgb}{0.0,0.8,0.0}
\definecolor{lightblue}{rgb}{0.3,0.8,1.0}
\definecolor{lightred}{rgb}{0.874,0.180,0.105}
\definecolor{gray}{rgb}{0.4,0.4,0.4}
\definecolor{lightgray}{rgb}{0.8,0.8,0.8}
\definecolor{shadecolor}{rgb}{0.9,0.9,0.9}

\title{Understanding variable importances\\[1ex]
 in forests of randomized trees}
\author{{\small Gilles Louppe, Louis Wehenkel, Antonio Sutera and Pierre Geurts {\it [Sun88]}}}
\date{}


\begin{document}

%% Slide 1 ==================================================================

\begin{frame}

\vspace{-1cm}

\begin{center}
\scalebox{0.43}{
    \begin{pspicture}(30,10)
    % Grid
    %\psgrid[subgriddiv=1,griddots=10,gridlabels=7pt]
    % Tree
        % Level 0
        \psline[linewidth=2pt,linecolor=lightgray]{->}(5,9.5)(3,8.5)
        \psline[linewidth=2pt,linecolor=lightgray]{->}(5,9.5)(7,8.5)
        \pscircle[fillstyle=solid,fillcolor=white,linewidth=2pt,linecolor=lightgray](5,9.5){0.5}
        % Level 1
        \psline[linewidth=2pt,linecolor=lightgray]{->}(3,8)(2,6.5)
        \psline[linewidth=2pt,linecolor=lightgray]{->}(3,8)(4,6.5)
        \pscircle[fillstyle=solid,fillcolor=white,linewidth=2pt,linecolor=lightgray](3,8){0.5}
        \psline[linewidth=2pt,linecolor=lightgray]{->}(7,8)(6,6.5)
        \psline[linewidth=2pt,linecolor=lightgray]{->}(7,8)(8,6.5)
        \pscircle[fillstyle=solid,fillcolor=white,linewidth=2pt,linecolor=lightgray](7,8){0.5}
        % Level 2
        \psline[linewidth=2pt,linecolor=lightgray]{->}(2,6)(1.5,4.5)
        \psline[linewidth=2pt,linecolor=lightgray]{->}(2,6)(2.5,4.5)
        \pscircle[fillstyle=solid,fillcolor=white,linewidth=2pt,linecolor=lightgray](2,6){0.5}
        \psline[linewidth=2pt,linecolor=lightgray]{->}(4,6)(3.5,4.5)
        \psline[linewidth=2pt,linecolor=lightgray]{->}(4,6)(4.5,4.5)
        \pscircle[fillstyle=solid,fillcolor=white,linewidth=2pt,linecolor=lightgray](4,6){0.5}
        \psline[linewidth=2pt,linecolor=lightgray]{->}(6,6)(5.5,4.5)
        \psline[linewidth=2pt,linecolor=lightgray]{->}(6,6)(6.5,4.5)
        \pscircle[fillstyle=solid,fillcolor=white,linewidth=2pt,linecolor=lightgray](6,6){0.5}
        \psline[linewidth=2pt,linecolor=lightgray]{->}(8,6)(7.5,4.5)
        \psline[linewidth=2pt,linecolor=lightgray]{->}(8,6)(8.5,4.5)
        \pscircle[fillstyle=solid,fillcolor=white,linewidth=2pt,linecolor=lightgray](8,6){0.5}
    % Tree
        % Level 0
        \psline[linewidth=2pt,linecolor=lightgray]{->}(13,9.5)(11,8.5)
        \psline[linewidth=2pt,linecolor=lightgray]{->}(13,9.5)(15,8.5)
        \pscircle[fillstyle=solid,fillcolor=white,linewidth=2pt,linecolor=lightgray](13,9.5){0.5}
        % Level 1
        \psline[linewidth=2pt,linecolor=lightgray]{->}(11,8)(10,6.5)
        \psline[linewidth=2pt,linecolor=lightgray]{->}(11,8)(12,6.5)
        \pscircle[fillstyle=solid,fillcolor=white,linewidth=2pt,linecolor=lightgray](11,8){0.5}
        \psline[linewidth=2pt,linecolor=lightgray]{->}(15,8)(14,6.5)
        \psline[linewidth=2pt,linecolor=lightgray]{->}(15,8)(16,6.5)
        \pscircle[fillstyle=solid,fillcolor=white,linewidth=2pt,linecolor=lightgray](15,8){0.5}
        % Level 2
        \psline[linewidth=2pt,linecolor=lightgray]{->}(10,6)(9.5,4.5)
        \psline[linewidth=2pt,linecolor=lightgray]{->}(10,6)(10.5,4.5)
        \pscircle[fillstyle=solid,fillcolor=white,linewidth=2pt,linecolor=lightgray](10,6){0.5}
        \psline[linewidth=2pt,linecolor=lightgray]{->}(12,6)(11.5,4.5)
        \psline[linewidth=2pt,linecolor=lightgray]{->}(12,6)(12.5,4.5)
        \pscircle[fillstyle=solid,fillcolor=white,linewidth=2pt,linecolor=lightgray](12,6){0.5}
        \psline[linewidth=2pt,linecolor=lightgray]{->}(14,6)(13.5,4.5)
        \psline[linewidth=2pt,linecolor=lightgray]{->}(14,6)(14.5,4.5)
        \pscircle[fillstyle=solid,fillcolor=white,linewidth=2pt,linecolor=lightgray](14,6){0.5}
        \psline[linewidth=2pt,linecolor=lightgray]{->}(16,6)(15.5,4.5)
        \psline[linewidth=2pt,linecolor=lightgray]{->}(16,6)(16.5,4.5)
        \pscircle[fillstyle=solid,fillcolor=white,linewidth=2pt,linecolor=lightgray](16,6){0.5}
    % Tree
        % Level 0
        \psline[linewidth=2pt,linecolor=lightgray]{->}(23,9.5)(21,8.5)
        \psline[linewidth=2pt,linecolor=lightgray]{->}(23,9.5)(25,8.5)
        \pscircle[fillstyle=solid,fillcolor=white,linewidth=2pt,linecolor=lightgray](23,9.5){0.5}
        % Level 1
        \psline[linewidth=2pt,linecolor=lightgray]{->}(21,8)(20,6.5)
        \psline[linewidth=2pt,linecolor=lightgray]{->}(21,8)(22,6.5)
        \pscircle[fillstyle=solid,fillcolor=white,linewidth=2pt,linecolor=lightgray](21,8){0.5}
        \psline[linewidth=2pt,linecolor=lightgray]{->}(25,8)(24,6.5)
        \psline[linewidth=2pt,linecolor=lightgray]{->}(25,8)(26,6.5)
        \pscircle[fillstyle=solid,fillcolor=white,linewidth=2pt,linecolor=lightgray](25,8){0.5}
        % Level 2
        \psline[linewidth=2pt,linecolor=lightgray]{->}(20,6)(19.5,4.5)
        \psline[linewidth=2pt,linecolor=lightgray]{->}(20,6)(20.5,4.5)
        \pscircle[fillstyle=solid,fillcolor=white,linewidth=2pt,linecolor=lightgray](20,6){0.5}
        \psline[linewidth=2pt,linecolor=lightgray]{->}(22,6)(21.5,4.5)
        \psline[linewidth=2pt,linecolor=lightgray]{->}(22,6)(22.5,4.5)
        \pscircle[fillstyle=solid,fillcolor=white,linewidth=2pt,linecolor=lightgray](22,6){0.5}
        \psline[linewidth=2pt,linecolor=lightgray]{->}(24,6)(23.5,4.5)
        \psline[linewidth=2pt,linecolor=lightgray]{->}(24,6)(24.5,4.5)
        \pscircle[fillstyle=solid,fillcolor=white,linewidth=2pt,linecolor=lightgray](24,6){0.5}
        \psline[linewidth=2pt,linecolor=lightgray]{->}(26,6)(25.5,4.5)
        \psline[linewidth=2pt,linecolor=lightgray]{->}(26,6)(26.5,4.5)
        \pscircle[fillstyle=solid,fillcolor=white,linewidth=2pt,linecolor=lightgray](26,6){0.5}
    \rput(18,7){...}
    \end{pspicture}
}

\vspace{1cm}

\textbf{Random Forests} are a well-tested, efficient and versatile tool.

Yet, they are {\color{red} still not fully \textbf{theoretically} understood}.

\end{center}

\end{frame}


%% Slide 2 ==================================================================

\setbeamertemplate{headline}{}
\begin{frame}

\vspace{-3.75cm}

\begin{center}
\scalebox{0.43}{
    \begin{pspicture}(30,10)
    % Grid
    %\psgrid[subgriddiv=1,griddots=10,gridlabels=7pt]
    % Tree
        % Level 0
        \psline[linewidth=2pt,linecolor=lightgray]{->}(5,9.5)(3,8.5)
        \psline[linewidth=2pt,linecolor=lightblue]{->}(5,9.5)(7,8.5)
        \pscircle[fillstyle=solid,fillcolor=white,linewidth=2pt,linecolor=lightblue](5,9.5){0.5}
        % Level 1
        \psline[linewidth=2pt,linecolor=lightgray]{->}(3,8)(2,6.5)
        \psline[linewidth=2pt,linecolor=lightgray]{->}(3,8)(4,6.5)
        \pscircle[fillstyle=solid,fillcolor=white,linewidth=2pt,linecolor=lightgray](3,8){0.5}
        \psline[linewidth=2pt,linecolor=lightblue]{->}(7,8)(6,6.5)
        \psline[linewidth=2pt,linecolor=lightgray]{->}(7,8)(8,6.5)
        \pscircle[fillstyle=solid,fillcolor=white,linewidth=2pt,linecolor=lightblue](7,8){0.5}
        % Level 2
        \psline[linewidth=2pt,linecolor=lightgray]{->}(2,6)(1.5,4.5)
        \psline[linewidth=2pt,linecolor=lightgray]{->}(2,6)(2.5,4.5)
        \pscircle[fillstyle=solid,fillcolor=white,linewidth=2pt,linecolor=lightgray](2,6){0.5}
        \psline[linewidth=2pt,linecolor=lightgray]{->}(4,6)(3.5,4.5)
        \psline[linewidth=2pt,linecolor=lightgray]{->}(4,6)(4.5,4.5)
        \pscircle[fillstyle=solid,fillcolor=white,linewidth=2pt,linecolor=lightgray](4,6){0.5}
        \psline[linewidth=2pt,linecolor=blue]{->}(6,6)(5.5,4.5)
        \psline[linewidth=2pt,linecolor=blue]{->}(6,6)(6.5,4.5)
        \pscircle[fillstyle=solid,fillcolor=white,linewidth=2pt,linecolor=blue](6,6){0.5}
        \psline[linewidth=2pt,linecolor=lightgray]{->}(8,6)(7.5,4.5)
        \psline[linewidth=2pt,linecolor=lightgray]{->}(8,6)(8.5,4.5)
        \pscircle[fillstyle=solid,fillcolor=white,linewidth=2pt,linecolor=lightgray](8,6){0.5}
    % Tree
        % Level 0
        \psline[linewidth=2pt,linecolor=lightblue]{->}(13,9.5)(11,8.5)
        \psline[linewidth=2pt,linecolor=lightgray]{->}(13,9.5)(15,8.5)
        \pscircle[fillstyle=solid,fillcolor=white,linewidth=2pt,linecolor=lightblue](13,9.5){0.5}
        % Level 1
        \psline[linewidth=2pt,linecolor=blue]{->}(11,8)(10,6.5)
        \psline[linewidth=2pt,linecolor=blue]{->}(11,8)(12,6.5)
        \pscircle[fillstyle=solid,fillcolor=white,linewidth=2pt,linecolor=blue](11,8){0.5}
        \psline[linewidth=2pt,linecolor=lightgray]{->}(15,8)(14,6.5)
        \psline[linewidth=2pt,linecolor=lightgray]{->}(15,8)(16,6.5)
        \pscircle[fillstyle=solid,fillcolor=white,linewidth=2pt,linecolor=lightgray](15,8){0.5}
        % Level 2
        \psline[linewidth=2pt,linecolor=lightgray]{->}(10,6)(9.5,4.5)
        \psline[linewidth=2pt,linecolor=lightgray]{->}(10,6)(10.5,4.5)
        \pscircle[fillstyle=solid,fillcolor=white,linewidth=2pt,linecolor=lightgray](10,6){0.5}
        \psline[linewidth=2pt,linecolor=lightgray]{->}(12,6)(11.5,4.5)
        \psline[linewidth=2pt,linecolor=lightgray]{->}(12,6)(12.5,4.5)
        \pscircle[fillstyle=solid,fillcolor=white,linewidth=2pt,linecolor=lightgray](12,6){0.5}
        \psline[linewidth=2pt,linecolor=lightgray]{->}(14,6)(13.5,4.5)
        \psline[linewidth=2pt,linecolor=lightgray]{->}(14,6)(14.5,4.5)
        \pscircle[fillstyle=solid,fillcolor=white,linewidth=2pt,linecolor=lightgray](14,6){0.5}
        \psline[linewidth=2pt,linecolor=lightgray]{->}(16,6)(15.5,4.5)
        \psline[linewidth=2pt,linecolor=lightgray]{->}(16,6)(16.5,4.5)
        \pscircle[fillstyle=solid,fillcolor=white,linewidth=2pt,linecolor=lightgray](16,6){0.5}
    % Tree
        % Level 0
        \psline[linewidth=2pt,linecolor=lightblue]{->}(23,9.5)(21,8.5)
        \psline[linewidth=2pt,linecolor=lightblue]{->}(23,9.5)(25,8.5)
        \pscircle[fillstyle=solid,fillcolor=white,linewidth=2pt,linecolor=lightblue](23,9.5){0.5}
        % Level 1
        \psline[linewidth=2pt,linecolor=lightgray]{->}(21,8)(20,6.5)
        \psline[linewidth=2pt,linecolor=lightblue]{->}(21,8)(22,6.5)
        \pscircle[fillstyle=solid,fillcolor=white,linewidth=2pt,linecolor=lightblue](21,8){0.5}
        \psline[linewidth=2pt,linecolor=blue]{->}(25,8)(24,6.5)
        \psline[linewidth=2pt,linecolor=blue]{->}(25,8)(26,6.5)
        \pscircle[fillstyle=solid,fillcolor=white,linewidth=2pt,linecolor=blue](25,8){0.5}
        % Level 2
        \psline[linewidth=2pt,linecolor=lightgray]{->}(20,6)(19.5,4.5)
        \psline[linewidth=2pt,linecolor=lightgray]{->}(20,6)(20.5,4.5)
        \pscircle[fillstyle=solid,fillcolor=white,linewidth=2pt,linecolor=lightgray](20,6){0.5}
        \psline[linewidth=2pt,linecolor=blue]{->}(22,6)(21.5,4.5)
        \psline[linewidth=2pt,linecolor=blue]{->}(22,6)(22.5,4.5)
        \pscircle[fillstyle=solid,fillcolor=white,linewidth=2pt,linecolor=blue](22,6){0.5}
        \psline[linewidth=2pt,linecolor=lightgray]{->}(24,6)(23.5,4.5)
        \psline[linewidth=2pt,linecolor=lightgray]{->}(24,6)(24.5,4.5)
        \pscircle[fillstyle=solid,fillcolor=white,linewidth=2pt,linecolor=lightgray](24,6){0.5}
        \psline[linewidth=2pt,linecolor=lightgray]{->}(26,6)(25.5,4.5)
        \psline[linewidth=2pt,linecolor=lightgray]{->}(26,6)(26.5,4.5)
        \pscircle[fillstyle=solid,fillcolor=white,linewidth=2pt,linecolor=lightgray](26,6){0.5}
    \rput(18,7){...}
    \end{pspicture}
}

\end{center}

\textbf{Variable importances} were first proposed {\color{red} as a heuristic}
to assess the influence of input variables.

\begin{align*}\label{eq:mdi}
Imp(X_m) &= \frac{1}{N_T} \sum_{T} \sum_{t \in T:v(t) = X_m} p(t) \Delta i(t)
\end{align*}

In the case of totally randomized trees, variable importances actually show
{\color{lightgreen} sound and desirable \textbf{theoretical} properties}.

\end{frame}


%% Slide 3 ==================================================================

\begin{frame}

\vspace{-2.5cm}

{\color{green} \cmark} Variable importances provide \textbf{a three-level decomposition of the
information jointly provided by all the input variables about the output},
accounting for all interaction terms in a fair and exhaustive way.

\vspace{-0.5cm}

\begin{flalign*}
\text{\textbf{Thm.~1:}} &\quad Imp(X_m)=\underbrace{\sum_{k=0}^{p-1} \frac{1}{C_p^k} \frac{1}{p-k}}_{{\color{lightblue} \substack{\text{ii) Decomposition along}\\
                                                                                                 \text{the degrees $k$ of interaction}\\
                                                                                                 \text{with the other variables}}}}
       \underbrace{\sum_{B \in {\cal P}_k(V^{-m})} I(X_m;Y|B)}_{{\color{lightred} \substack{\text{iii) Decomposition along all}\\
                                                                                            \text{interaction terms $B$}\\
                                                                                            \text{of a given degree $k$}}}}\\
\text{\textbf{Thm.~2:}} &\quad \underbrace{\sum_{m=1}^{p}Imp(X_m)}_{{\color{lightgreen} \substack{\text{i) Decomposition in terms of}\\
                                                                   \text{the MDI importance of}\\
                                                                   \text{each input variable}}}} =
\underbrace{I(X_{1}, \ldots, X_{p} ; Y)}_{\substack{\text{Information jointly provided}\\
                                                    \text{by all input variables}\\
                                                    \text{about the output}}}
\end{flalign*}

\end{frame}


%% Slide 4 ==================================================================

\begin{frame}

\vspace{-3cm}

{\color{green} \cmark} Variable importances \textbf{depend only on the relevant variables}.

\vspace{1cm}

\textbf{Thm.~3:} A variable is irrelevant if and only if its importance is 0.

\vspace{1cm}

\textbf{Thm.~5:} The importance of a relevant variable is insensitive to the addition or the removal of irrelevant variables.

\end{frame}

\end{document}
