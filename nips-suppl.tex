% Packages =====================================================================

\documentclass{article}
\usepackage{nips13submit_e}
\usepackage{times}
\usepackage[utf8]{inputenc}
\usepackage[authoryear]{natbib}
\usepackage{amsmath}
\usepackage{amsthm}
\usepackage{amssymb}
\usepackage{xcolor}
\usepackage{auto-pst-pdf}
\usepackage{pst-plot}
\usepackage{floatrow}
\usepackage[font=small,labelfont=bf]{caption}



% Various settings =============================================================

\newtheorem{theorem}{Theorem}
\newtheorem{lemma}[theorem]{Lemma}
\newtheorem{proposition}[theorem]{Proposition}
\newtheorem{corollary}[theorem]{Corollary}

\usepackage{xr}
\externaldocument{nips}

\newcommand{\lwh}[1]{\textcolor{blue}{#1}}



% Title ========================================================================

\title{\textbf{Understanding variable importances\\
in forests of randomized trees\\
\textit{Supplementary materials}}}

\author{
Gilles Louppe$^{a}$, Louis Wehenkel$^{a}$, Antonio Sutera$^{b}$ and Pierre Geurts$^{a}$ \\
Dept. of EE \& CS, University of Liège, Belgium \\
$^{a}$: \texttt{\{g.louppe, l.wehenkel, p.geurts\}@ulg.ac.be}\\
$^{b}$: \texttt{antonio.sutera@student.ulg.ac.be}\\
% \And
% Louis Wehenkel \\
% University of Liège, Belgium \\
% \texttt{l.wehenkel@ulg.ac.be} \\
% % \And
% % Antonio Sutera \\
% % University of Liège, Belgium \\
% % \texttt{antonio.sutera@student.ulg.ac.be} \\
% \And
% Pierre Geurts \\
% University of Liège, Belgium \\
% \texttt{p.geurts@ulg.ac.be} \\
}


\newcommand{\fix}{\marginpar{FIX}}
\newcommand{\new}{\marginpar{NEW}}

\nipsfinalcopy % Uncomment for camera-ready version

\begin{document}

\maketitle
\allowdisplaybreaks



% Appendix =====================================================================

\appendix


% ------------------------------------------------------------------------------

\section{Notations, and definitions of entropies and mutual information}

To be self-contained, we first recall several definitions from information theory (see \cite{cover2012elements}, for further properties).

We suppose that we are given a probability space $(\Omega, {\cal E}, \mathbb{P})$ and  consider random variables
defined on it taking a finite number of possible values. We use upper case letters to denote such random variables (e.g. $X, Y, Z, W \ldots$)  and calligraphic letters (e.g. $\cal X, Y, Z, W \ldots$) to denote their image sets (of finite cardinality), and lower case letters (e.g. $x, y, z, w \ldots$) to denote one of their possible values.
For a (finite) set of (finite) random variables $ X = \{X_{1}, \ldots , X_{i}\}$, we denote by $P_{X}(x) = P_{X}(x_{1}, \ldots , x_{i})$ the probability $\mathbb{P}(\{ \omega \in \Omega \mid  \forall \ell : 1, \ldots, i: X_{\ell}(\omega) =x_{\ell}\})$, and by ${\cal X} = {\cal X}_{1} \times \cdots \times {\cal X}_{i}$ the set of joint configurations of these random variables. Given two sets of random variables, $X = \{X_{1}, \ldots , X_{i}\}$ and $Y=\{Y_{1}, \ldots , Y_{j}\}$, we denote by $P_{X \mid Y}(x \mid y) = {P_{X, Y} (x,  y)}/ {P_{Y}(y)}$ the conditional density of $X$ with respect to $Y$.\footnote{To avoid problems, we suppose that all probabilities are strictly positive, without fundamental limitation.}

With these notations, the joint (Shannon) entropy of a set of random variables $X =\{X_{1}, \ldots , X_{i}\}$ is thus defined by
\begin{equation*}
H(X)  = - \sum_{x \in {\cal X}}P_{X} (x)\log_{2}P_{X }(x),
\end{equation*}
while the mean conditional entropy of a set of random variables $X = \{X_{1}, \ldots , X_{i}\}$, given the values of another set of random variables $Y=\{Y_{1}, \ldots , Y_{j}\}$ is defined by
\begin{equation*}
H(X\mid Y) = - \sum_{x \in {\cal X}} \sum_{y \in {\cal Y}} P_{X, Y} (x, y) \log_{2} P_{X \mid Y} (x  \mid y).
 \end{equation*}
The mutual information among the set of random variables $X =\{X_{1}, \ldots , X_{i}\}$ and the set of random variables $Y=\{Y_{1}, \ldots , Y_{j}\}$ is defined by
 \begin{eqnarray*}
 I(X; Y) & = &- \sum_{x \in {\cal X}} \sum_{y \in {\cal Y}} P_{X, Y} (x, y) \log_{2} \frac{P_{X}(x) P_{Y}(y)}{P_{X,Y}(x,y)} \\
 & = & H(X) - H(X \mid Y) \\
 & = &  H(Y) - H(Y \mid X).
 \end{eqnarray*}
 The mean conditional mutual information among the set of random variables $X =\{X_{1}, \ldots , X_{k}\}$ and the set of random variables $Y=\{Y_{1}, \ldots , Y_{j}\}$, given the values of a third set of random variables $Z=\{Z_{1}, \ldots , Z_{i}\}$, is defined by
 \begin{eqnarray*}
 I(X; Y \mid Z) &= &H(X \mid Z) - H(X \mid Y, Z)\\
 & = & H(Y \mid Z) - H(Y \mid X, Z)\\
& = & - \sum_{x \in {\cal X}} \sum_{y \in {\cal Y}} \sum_{z \in {\cal Z}} P_{X, Y, Z} (x, y, z) \log_{2} \frac{P_{X \mid Z}(x \mid z) P_{Y\mid Z}(y \mid z)}{P_{X,Y \mid Z}(x,y \mid z)} .
\end{eqnarray*}

We also recall the chaining rule
\begin{equation*}
I(X, Z ; Y \mid W ) = I(X; Y \mid W  ) + I( Z ; Y \mid W, X),
\end{equation*}
and the symmetry of the (conditional) mutual information among sets of random variables
\begin{equation*}
I(X ; Y \mid Z) = I(Y ;  X  \mid Z).
\end{equation*}


\section{Proof of Theorem~\ref{thm:imp}}
\label{app:thm:imp}

\begin{proof}
Let us define $p(t)$ as the proportion $N_t / N$ of samples from ${\cal L}$
reaching node $t$ and $p(j|t)$ as the proportion $N_{jt} / N_t$ of samples
of class $j$ in $t$. By expanding Equation~\ref{eq:deltaimpurity} into
Equation~\ref{eq:mdi} and using the entropy $H(Y|t) = -\sum_j p(j|t) \log_2(p(j|t))$
as impurity measure $i(t)$, Equation~\ref{eq:mdi} can be rewritten in terms of
mutual information:
\begin{equation*}
Imp(X_m) = \frac{1}{N_T} \sum_{T} \sum_{t \in T:v(s_t) = X_m} p(t) I(Y;X_m|t)
\end{equation*}

As the size $N$ of the training sample grows to infinity,
$p(t)$ becomes the (exact) probability (according to $P(X_1,\ldots,X_p,Y)$) that
an object reaches node $t$, i.e., a probability $P(B(t)=b(t))$ where $B(t)=(X_{i_1}, ..., X_{i_{k}})$ is
the subset of $k$ variables tested in the branch from the root node to the parent of $t$ and
$b(t)$ is the vector of values of these variables. As the the number $N_T$ of
totally randomized trees also grows to infinity, the importance of a variable
$X_m$ can then be written:
\begin{equation*}
Imp(X_m)=\sum_{B\subseteq V^{-m}} \sum_{b\in {\cal X}_{i_1}\times ... \times {\cal X}_{i_{k}} } \alpha(B,b,X_m,p) P(B=b) I(Y;X_m|B=b),
\end{equation*}
where $b$ is a set of values for the variables in $B$ and $\alpha(B,b,X_m,p)$ is
the probability that a node $t$ (at depth $k$) in a totally randomized tree
tests the variable $X_m$ and is such that $B(t)=B$ and $b(t)=b$.

Let us compute $\alpha(B,b,X_m,p)$. First, let us consider the probability that
a node $t$ tests the variable $X_m$ and is such that the branch leading to $t$
follows a path defined, in that particular order, by all $k$ variables $X_{i_1},
..., X_{i_{k}} \in B$ and their corresponding values in $b$. The probability of
that branch is the probability of picking (uniformly at random)  $X_{i_1}$  at
the root node times the probability of testing, in that order, the remaining
$X_{i_2}, ..., X_{i_{k}}$ variables in the sub-tree corresponding to the value
$x_{i_1}$ of $X_{i_1}$ defined in $b$. Note that, by construction, it is certain
that this particular sub-tree exists since the root node is split into $|{\cal
X}_{i_1}|$ sub-trees.  Then, the
probability of testing $X_m$ at the end of this branch is the probability of
picking $X_m$ among the remaining $p-k$ variables. By recursion, we thus have:
$$\frac{1}{p} \frac{1}{p-1} ... \frac{1}{p-k+1} \frac{1}{p-k} =
\frac{(p-k)!}{p!} \frac{1}{p-k}$$

Since the order along which the variables appear in the branch is of no
importance, $\alpha(B,b,X_m,p)$ actually includes all $k!$ equiprobable ways of
building a branch composed of the variables and values in $B$ and $b$. Then,
since a tree may at most contain a single such branch, whatever the order of the
tests, the probabilities may be added up and it comes:
\begin{equation*}
\alpha(B,b,X_m,p)=k! \frac{(p-k)!}{p!} \frac{1}{p-k} = \frac{1}{C_p^k} \frac{1}{p-k}
\end{equation*}

From the above expression, it appears that $\alpha(B,b,X_m,p)$ depends only on
the size $k$ of $B$ and on the number $p$ of variables. As such, by grouping in
the previous equation of $Imp(X_m)$ conditioning variable subsets $B$ according
to their sizes and using the definition of conditional mutual information,
$\alpha$ can be factored out, hence leading to the form foretold by Theorem~\ref{thm:imp}:
\begin{equation*}
Imp(X_m)=\sum_{k=0}^{p-1} \frac{1}{C^k_p} \frac{1}{p-k} \sum_{B \in {\cal P}_k(V^{-m})} I(X_m;Y|B).
\end{equation*}
\end{proof}



% ------------------------------------------------------------------------------


\section{Proof of Theorem~\ref{thm:sum-of-imp}}
\label{app:thm:sum-of-imp}

\begin{proof}
For any tree $T$, we have that the sum of all
importances estimated by using an infinitely large sample $\cal L$ (or
equivalently, by assuming perfect knowledge of the joint distribution $P(X_1,
..., X_p, Y)$) is equal to $H(Y) - \sum_{l \in \ell(T)} p(l) H(Y|b(l))$, where
$\ell(T)$ denotes the set of all leaves of $T$, and where $b(l)$  denotes the
joint configuration of all input variables leading to leaf $l$. This is true because the impurities of all test nodes intervening in the computation of the variable importances, except the impurity $H(Y)$ at the root node of the tree, cancel each other when summing up the importances.

Since, when the tree is fully developed,
$\sum_{l \in \ell(T)} p(l) H(Y|b(l))$ is obviously equal to the mean conditional entropy
$H(Y | X_{1}, \ldots, X_{p})$ of $Y$ given all input variables, this implies
that for any fully developed tree we have that the sum of variable importances
is equal to $I(X_{1}, \ldots, X_{p} ; Y)$, and so this relation also holds when
averaging over an infinite ensemble of totally randomized trees.
\end{proof}


% ------------------------------------------------------------------------------

\section{Proof of Theorem~\ref{thm:irrelevant}}
\label{app:thm:irrelevant}

\begin{proof}
The proof directly results from the definition of irrelevance. If $X_i$ is
irrelevant with respect to $V$, then $I(X_i;Y|B)$ is zero for all $B \subseteq
V^{-i} \subset V$ and Equation~\ref{eqn:imp-full} reduces to $0$. Also, since
$I(X_i;Y|B)$ is non-negative for any $B$, $Imp(X_i)$ is zero if and only if all
its $I(X_i;Y|B)$ terms are zero. Since $Imp(X_i)$ includes all $I(X_i;Y|B)$
terms for $B \subseteq V^{-i}$, and since all of them are therefore null if
$Imp(X_i)=0$,  $X_i$ is thus, by definition, irrelevant with respect to
$V^{-i}$. $X_i$ is then also trivially irrelevant with respect to $V=V^{-i} \cup
\{X_i\}$ since $I(X_i;Y|B\cup\{X_i\})=0$ for any $B$.
\end{proof}


% ------------------------------------------------------------------------------

\section{Proof of Lemma~\ref{lemma:adding-irrelevant}}
\label{app:lemma:adding-irrelevant}

\begin{proof}
Let $X_i \notin V$ be an irrelevant variable with respect to $V$. For $X_m \in
V$, $B \subseteq V^{-m}$, using the chain rules of mutual information, we have:
\begin{eqnarray*}
I(X_m, X_i;Y|B) &=& I(X_m;Y|B) + I(X_i;Y|B \cup \{X_m\}) \\
                &=& I(X_i;Y|B) + I(X_m;Y|B \cup \{X_i\})
\end{eqnarray*}
If $X_i$ is irrelevant with respect to $V$, i.e., such that $I(X_i;Y|B)=0$ for
all $B\subseteq V$, then $I(X_i;Y|B \cup \{X_m\})$ and $I(X_i;Y|B)$ both equal
$0$, leading to
\begin{eqnarray*}
I(X_m;Y|B \cup \{X_i\}) = I(X_m;Y|B)
\end{eqnarray*}

Then, from Theorem~\ref{thm:imp}, the importance of $X_m$ as computed with an
infinite ensemble of totally randomized trees built on $V\cup \{X_i\}$ can be
simplified to:

\begin{eqnarray*}
  Imp(X_m)&=&\sum_{k=0}^{p-1+1} \frac{1}{C_{p+1}^k} \frac{1}{p+1-k} \sum_{B \in {\cal P}_k(V^{-m} \cup \{X_i\})} I(X_m;Y|B)\\
          &=&\sum_{k=0}^{p} \frac{1}{C_{p+1}^k} \frac{1}{p+1-k} \left[ \sum_{B \in {\cal P}_k(V^{-m})} I(X_m;Y|B) + \sum_{B \in {\cal P}_{k-1}(V^{-m})} I(X_m;Y|B \cup \{X_i\})  \right] \\
          &=&\sum_{k=0}^{p-1} \frac{1}{C_{p+1}^k} \frac{1}{p+1-k} \sum_{B \in {\cal P}_k(V^{-m})} I(X_m;Y|B) + \\
          & &\hookrightarrow \sum_{k=1}^{p} \frac{1}{C_{p+1}^k} \frac{1}{p+1-k} \sum_{B \in {\cal P}_{k-1}(V^{-m})} I(X_m;Y|B)\\
          &=&\sum_{k=0}^{p-1} \frac{1}{C_{p+1}^k} \frac{1}{p+1-k} \sum_{B \in {\cal P}_k(V^{-m})} I(X_m;Y|B) + \\
          & &\hookrightarrow \sum_{k'=0}^{p-1} \frac{1}{C_{p+1}^{k'+1}} \frac{1}{p+1-k'-1} \sum_{B \in {\cal P}_{k'}(V^{-m})} I(X_m;Y|B)\\
          &=&\sum_{k=0}^{p-1} \left[ \frac{1}{C_{p+1}^k} \frac{1}{p+1-k} + \frac{1}{C_{p+1}^{k+1}} \frac{1}{p-k} \right] \sum_{B \in {\cal P}_k(V^{-m})} I(X_m;Y|B)\\
          &=&\sum_{k=0}^{p-1} \frac{1}{C^k_p} \frac{1}{p-k} \sum_{B \in {\cal P}_k(V^{-m})} I(X_m;Y|B)\\
\end{eqnarray*}

The last line above exactly corresponds to the importance of $X_m$ as computed
with an infinite ensemble of totally randomized trees built on $V$, which proves
Lemma~\ref{lemma:adding-irrelevant}.
\end{proof}


% ------------------------------------------------------------------------------

\section{Proof of Theorem~\ref{thm:relevant}}
\label{app:thm:relevant}

\begin{proof}
Let us assume that $V_R$ contains $r \leq p$ relevant variables. If an infinite
ensemble of totally randomized trees were to be built directly on those $r$
variables then, from Theorem~\ref{thm:imp}, the importance of a relevant
variable $X_m$ would be:
\begin{equation*}%\label{eqn:imp-rel}
  Imp(X_m)=\sum_{l=0}^{r-1} \frac{1}{C_r^l} \frac{1}{r-l} \sum_{B \in {\cal P}_l(V_R^{-m})} I(X_m;Y|B)
\end{equation*}

Let $X_i \in V \setminus V_R$ be one of the $p-r$ irrelevant variables in $V$
with respect to $V$. Since $X_i$ is also irrelevant with respect to $V_R$, using
Lemma~\ref{lemma:adding-irrelevant}, the importance of $X_m$ when the ensemble
is built on $V_R \cup \{X_i\}$ is the same as the one computed on $V_R$ only
(i.e., as computed by the equation above). Using the same argument,
adding a second irrelevant variable $X_{i'}$ with respect to $V$ -- and therefore
also with respect to $V_R \cup \{X_i\}$ -- and building an ensemble of totally
randomized trees on $V_R \cup \{X_i\} \cup \{X_{i'}\}$ will yield importances
that are the same as those computed on $V_R \cup \{X_i\}$, which are themselves
the same as those computed by an ensemble built on $V_R$. By induction, adding
all $p-r$ irrelevant variables has therefore no effect on the importance of
$X_m$, which means that:

\begin{equation*}%\label{eqn:imp-rel-equiv}
  \begin{aligned}
  Imp(X_m)&=\sum_{k=0}^{p-1} \frac{1}{C_p^k} \frac{1}{p-k} \sum_{B \in {\cal P}_k(V^{-m})} I(X_m;Y|B)\\
          &=\sum_{l=0}^{r-1} \frac{1}{C_r^l} \frac{1}{r-l} \sum_{B \in {\cal P}_l(V_R^{-m})} I(X_m;Y|B)\\
  \end{aligned}
\end{equation*}
\end{proof}

Intuitively, the independence with respect to irrelevant variables can be partly
attributed to the fact that splitting at $t$ on some irrelevant variable $X_i$
should only dilute the local importance $p(t) \Delta i(t)$ of a relevant
variable $X_m$ into the children $t_L$ and $t_R$, but not affect the total sum.
For instance, if $X_m$ was to be used at $t$, then the local importance would be
proportional to $p(t)$. By contrast, if $X_i$ was to be used at $t$ and $X_m$ at
$t_L$ and $t_R$, then the sum of the local importances for $X_m$ would be
proportional to $p(t_L) + p(t_R)=p(t)$, which does not change anything.
Similarly, one can recursively invoke the same argument if $X_m$ was to be used
deeper in $t_L$ or $t_R$.



% ------------------------------------------------------------------------------

\section{Proof of Proposition~\ref{proposition:pruning}}
\label{app:proposition:pruning}

\begin{proof}
The proof of Theorem~\ref{thm:imp} can be directly adapted to prove
Proposition~\ref{proposition:pruning}. If the recursive procedure is stopped at
depth $q$, then it means that $B(t)$ may include up to $q-1$ variables, which is
strictly equivalent to summing from $k=0$ to $q-1$ in the outer sum of
Equation~\ref{eqn:imp-full}.
\end{proof}


\section{Proof of Proposition~\ref{proposition:imp-subspaces}}
\label{app:proposition:imp-subspaces}

\begin{proof}
Let us define a random subspace of size $q$ as a random subset $V_S \subseteq V$
such that $|V_S|=q$. By replacing $p$ with $q$ in Equation~\ref{eqn:imp-full}
(since each tree is built on $q$ variables) and adjusting by the probability
$$\frac{C^{q-k-1}_{p-k-1}}{C^q_p}$$ of having selected $X_m$ and the $k$ variables in
the branch when drawing $V_S$ prior to the construction of the tree, it comes:

\begin{equation*}
Imp(X_m)=\sum_{k=0}^{q-1} \frac{C^{q-k-1}_{p-k-1}}{C^q_p} \frac{1}{C_q^k} \frac{1}{q-k} \sum_{B \in {\cal P}_k(V^{-m})} I(X_m;Y|B)
\end{equation*}

The multiplicative factor in the outer sum can then be simplified as follows:
\begin{eqnarray*}
\frac{C^{q-k-1}_{p-k-1}}{C^q_p} \frac{1}{C_q^k} \frac{1}{q-k} &=& \frac{\frac{(p-k-1)!}{(p-k)!(q-k-1)!}}{\frac{p!}{(p-q)!q!}} \frac{1}{C_q^k} \frac{1}{q-k}\\
&=& \frac{(p-k-1)!q!}{(q-k-1)!p!} \frac{1}{C_q^k} \frac{1}{q-k} \\
&=& \frac{q(q-1) ... (q-k)}{p(p-1) ... (p-k)} \frac{1}{C_q^k} \frac{1}{q-k}\\
&=& \frac{q(q-1) ... (q-k)}{p(p-1) ... (p-k)} \frac{k!(q-k)!}{q!} \frac{1}{q-k}\\
&=& \frac{1}{p(p-1) ... (p-k)} \frac{k!(q-k)!}{(q-k-1)!} \frac{1}{q-k}\\
&=& \frac{k!}{p(p-1) ... (p-k)} \\
&=& \frac{k!(p-k)!}{p!} \frac{1}{p-k} \\
&=& \frac{1}{C^k_p} \frac{1}{p-k} \\
\end{eqnarray*}
which  yields the same importance as in Proposition~\ref{proposition:pruning} and proves the
proposition.
\end{proof}



% ------------------------------------------------------------------------------

\section{Generalization to other impurity measures}
\label{app:generalization}

In this appendix, we show that most of our results can be carried over
to other impurity measures. We first sketch the proof that
Theorems~\ref{thm:imp}, \ref{thm:irrelevant} and \ref{thm:relevant}
hold generically, and then discuss the case of some common impurity
measures.

%\textcolor{red}{à retravailler}

\subsection{Generalization of Theorems~\ref{thm:imp}, \ref{thm:irrelevant} and \ref{thm:relevant}}

Let us consider a generic impurity measure $i(Y|t)$ and, by mimicking the
notation used for conditional mutual information, let us denote by $G(Y;X_m|t)$
the impurity decrease for a split on $X_m$ at node $t$:
\begin{equation*}%\label{eqn:impsample}
G(Y;X_m|t) = i(Y|t)-\sum_{x\in{\cal X}_m} p(t_x) i(Y|t_x),
\end{equation*}
where $t_{x}$ denotes the successor node of $t$ corresponding to value
$x$ of $X_m$. The importance score associated to a variable $X_m$ (see
Equation~\ref{eq:mdi}) is then rewritten:
\begin{equation*}\label{eq:mdigen}
Imp(X_m) = \frac{1}{N_T} \sum_{T} \sum_{t \in T:v(s_t) = X_m} p(t) G(Y;X_m|t).
\end{equation*}

As explained in the proof of Appendix~\ref{app:thm:imp}, conditioning
over a node $t$ is equivalent to conditioning over an event of the
form $B(t)=b(t)$, where $B(t)$ and $b(t)$ denote respectively the set
of variables tested in the branch from the root to $t$ and their
values in this branch. When the learning sample size $N$ grows to
infinity, this yields the following population based impurity decrease
at node $t$:
\begin{eqnarray*}
&&G(Y;X_m|B(t)=b(t))\nonumber \\
& = & i(Y|B(t)=b(t))-\sum_{x\in{\cal X}_m}
P(X_m=x|B(t)=b(t)) i(Y|B(t)=b(t),X_m=x).\nonumber \\\label{eqn:imppop}
\end{eqnarray*}
Again by analogy with conditional entropy and mutual
information\footnote{Note however that $G(Y;X_m|B)$ does not share all
  properties of conditional mutual information as for example $G(X_m;
  Y|B)$ might not be equal to $G(Y; X_m|B)$ or even be defined,
  depending on the impurity measure and the nature of the output
  $Y$.}, let us define $i(Y|B)$ and $G(Y;X_m|B)$ for some subset of
variables $B\subseteq V$ as follows:
\begin{eqnarray*}
i(Y|B)&=&\sum_{b} P(B=b) i(Y|B=b)\\
G(Y;X_m|B)&=& \sum_{b} P(B=b) G(Y;X_m|B=b)\\
&=&i(Y|B)-i(Y|B,X_m)
\end{eqnarray*}
where the sums run over all possible combinations $b$ of values for
the variables in $B$.

With these notations, the proof of Theorem~\ref{thm:imp} can be easily adapted to lead
to the following generalization of Equation~\ref{eqn:imp-full}:
\begin{equation}
Imp(X_m)=\sum_{k=0}^{p-1} \frac{1}{C_p^k} \frac{1}{p-k} \sum_{B \in
  {\cal P}_k(V^{-m})} G(Y;X_m|B).
\end{equation}
Note that this generalization is valid without any further specific constraints
on the impurity measure $i(Y|t)$.

Let us now define as {\it irrelevant to $Y$ with respect to $V$} a
variable $X_i$ for which, for all $B\subseteq V$,
$G(Y;X_i|B)=0$ (i.e. a variable that neither affects impurity
whatever the conditioning). From this definition, one can deduce the
following property of an irrelevant variable $X_i$ (for all
$B\subseteq V$ and $X_m\in V$):
$$G(Y;X_m|B\cup\{X_i\})=G(Y;X_m|B).$$

Indeed, by a simple application of previous definitions, we have:
\begin{eqnarray*}
&&G(Y;X_m|B)-G(Y;X_m|B\cup\{X_i\})\nonumber\\
& = & i(Y|B)-i(Y|B\cup\{X_m\})-i(Y|B\cup\{X_i\})+i(Y|B\cup\{X_i,X_m\})\\
& = & i(Y|B)-i(Y|B\cup\{X_i\})-i(Y|B\cup\{X_m\})+i(Y|B\cup\{X_i,X_m\})\\
& = & G(Y;X_i|B)-G(Y;X_i|B\cup\{X_m\})\\
& = & 0,
\end{eqnarray*}
where the last step is a consequence of the irrelevance of $X_i$.


%% \sum_b P(B=b) \left(
%% i(Y|B=b)-\sum_{x_m} P(X_m=x_m|B=b) i(Y|B=b,X_m=x_m)\right)\nonumber\\
%% &-& \sum_{b,x_i} P(B=b,X_i=x_i)\nonumber\\
%% && \left( i(Y|B=b,X_i=x_i)-\sum_{x_m}
%% P(X_m=x_m|B=b,X_i=x_i) i(Y|B=b,X_i=x_i,X_m=x_m)\right)\nonumber\\
%% & = & \sum_b P(B=b)\left(
%% i(Y|B=b)-\sum_{x_i} P(X_i=x_i|B=b) i(Y|B=b,X_i=x_i)\right)\nonumber\\
%% &-& \sum_{b,x_m} P(B=b,X_m=x_m)\nonumber\\
%% && \left( i(Y|B=b,X_m=x_m)-\sum_{x_i}
%% P(X_i=x_i|B=b) i(Y|B=b,X_i=x_i,X_m=x_m)\right)\nonumber\\
%% &=& G(Y;X_i|B)-G(Y;X_i|B\cup\{X_m\})\label{eqn:chain}
%%\end{eqnarray}
%}

%% By definition of an irrelevant variable, we have:
%% $$G(Y;X_i|B)=G(Y;X_i|B\cup\{X_m\})=0,$$
%% which, plugged in Equation~\ref{eqn:chain}, gives:
%% $$G(Y;X_m|B\cup\{X_i\})=G(Y;X_m|B).$$

Using this property, the proofs of Lemma~\ref{lemma:adding-irrelevant}
and Theorem~\ref{thm:relevant} in
Appendices~\ref{app:lemma:adding-irrelevant} and
\ref{app:thm:relevant} can be straightforwardly adapted, showing that,
in the general case also, the MDI importance of a variable is
invariant with respect to the removal or the addition of irrelevant
variables.

Given the general definition of irrelevance, all irrelevant variables
also get zero MDI importance but, without further constraints on the
impurity measure $i$, there is no guarantee that all relevant
variables (defined as all variables that are not irrelevant) will get
a non zero importance. This property, and in consequence
theorem~\ref{thm:irrelevant}, will be however satisfied as soon as the
impurity measure is such that $G(Y;X_m|B)\geq 0$ for all $X_m\in V$
and for all $B\subseteq V$.

\subsection{Common impurity measures}

Developments in the previous section show that all results in the
paper remain valid for any impurity measure leading to non negative
impurity decreases, provided that the definition of variable
irrelevance is adapted to this impurity measure. The choice of a
specific impurity measure should thus be guided by the meaning one
wants to associate to irrelevance.

Measuring impurity with Shannon entropy, i.e., taking $i(Y|t)=H(Y|t)$
and $i(Y|B=b)=H(Y|B=b)$, one gets back all results in the paper. Given the
properties of conditional mutual information, irrelevance for this
impurity measure strictly coincides with conditional independence: a
variable $X_i$ is irrelevant to $Y$ with respect to $V$ if and only if
$X_i\perp Y|B$ for all $B\subseteq V$.

A common alternative to Shannon entropy for growing classification
trees is Gini index (or entropy), which, in the finite and infinite
sample cases, is written:
\begin{eqnarray*}
i(Y|t)&=& -\sum_j p(j|t) (1-p(j|t))\\
i(Y|B=b)&=& -\sum_j P(Y=j|B=b) (1-P(Y=j|B=b)).
\end{eqnarray*}
Like Shannon entropy, this measure leads to non negative impurity
decreases and the corresponding notion of irrelevance is also directly
related to conditional independence.

The most common impurity measure for regression is variance, which, in
the finite and infinite sample cases, is written:
\begin{eqnarray*}
i(Y|t)&=&\frac{1}{N_t} \sum_{i\in t} (y_i-\frac{1}{N_t}
\sum_{i\in t} y_i)^2\\
i(Y|B=b)&=&E_{Y|B=b}\{(Y-E_{Y|B=b}\{Y\})^2\}.
\end{eqnarray*}
Variance can only decrease as a consequence of a split and therefore,
Theorem~\ref{thm:irrelevant} is also valid for this impurity measure,
meaning that only irrelevant variables will get a zero variance
reduction. Note however that with this impurity measure, irrelevance
is not directly related to conditional independence, as some variable
$X_i$ can be irrelevant in the sense of our definition and still
affects the distribution of output values.

%\clearpage
\bibliographystyle{apalike}
\bibliography{bib}
\vfill

\end{document}
